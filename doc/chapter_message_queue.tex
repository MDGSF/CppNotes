\chapter{消息队列}

\section{redis}

redis提供了发布订阅模式。所以可以借助redis在不同的进程之间通信。redis多用于服务
器上,不过redis是纯C语言编写的,所以很容易迁移到嵌入式上运行。

https://redis.io/topics/pubsub

\section{nanomsg}

nanomsg 是一个纯C语言编写的网络通信库,提供了几种常用的通信模式:

\begin{enumerate}
  \item Pipeline (A One-Way Pipe)
  \item Request/Reply (I ask, you answer)
  \item Pair (Two Way Radio)
  \item Pub/Sub (Topics \& Broadcast)
  \item Survey (Everybody Votes)
  \item Bus (Routing)
\end{enumerate}

\subsection{Pipeline}

https://nanomsg.org/gettingstarted/pipeline.html

Push作为客户端,Pull作为服务器端。
Push只发送消息。Pull只接收来自Push的消息。

\begin{verbatim}
   Client                       Server
+--------+                   +--------+
|  Push  +--------+--------->+  Pull  |
+--------+        |          +--------+
                  |
+--------+        |
|  Push  +--------+
+--------+
\end{verbatim}

\subsection{Request/Reply}

https://nanomsg.org/gettingstarted/reqrep.html

Request作为客户端,Response作为服务器端。
Response收到一个消息,就要回复一个消息。同步式的一问一答。

\begin{verbatim}
   Client                       Server
+--------+                   +--------+
|  Req   +<-------+--------->+  Rep   |
+--------+                   +--------+
\end{verbatim}

\subsection{Pair}

https://nanomsg.org/gettingstarted/pair.html

其中任意一个节点作为服务器端,另一个节点作为客户端。
两个端点都可以任意收发数据。就像普通的全双工通信一样。

\begin{verbatim}
   Client                       Server
+--------+                   +--------+
| node0  +<-------+--------->+ node1  |
+--------+                   +--------+
\end{verbatim}

\subsection{Pub/Sub}

https://nanomsg.org/gettingstarted/pubsub.html

Publish作为服务器端,可以有任意多个Subscribe。
Publish只负责发送数据,Subscribe只负责读取数据。
消息过滤实在Subscribe端处理的。

\begin{verbatim}
+--------+                    +--------+
| Client |                    | Client |
+---+----+                    +---+----+
    ^                             ^
    |        +----------+         |
    +--------+  Server  +---------+
             +----+-----+
                  |
                  |
             +----v---+
             | Client |
             +--------+
\end{verbatim}

\subsection{Survey}

https://nanomsg.org/gettingstarted/survey.html

Surveyor作为服务器端,可以有多个客户端。
由Surveyor发起投票,然后Surveyor会在一定的时间内等待来自客户端的消息。

\begin{verbatim}
+--------+                    +--------+
| Client |                    | Client |
+---+----+                    +---+----+
    ^                             ^
    |        +----------+         |
    +------->+  Server  +<--------+
             +----+-----+
                  ^
                  |
             +----v---+
             | Client |
             +--------+
\end{verbatim}

\subsection{Bus}

https://nanomsg.org/gettingstarted/bus.html

在Bus模式下,每个节点都是既作为服务器端又同时作为客户端,每个节点都互相连接。

\begin{verbatim}
+------+          +------+
| node <----------> node |
+---^--+          +--^---+
    |                |
    |                |
    |                |
+---v--+             |
| node <-------------+
+------+
\end{verbatim}

\section{libflow}

libflow是用C++写的网络通信库,支持发布订阅模式,底层支持TCP、UDP通信。

\section{dds}

dds 是基于共享内存的本地通信库,支持发布订阅模式。

\section{workflow}

https://github.com/sogou/workflow

workflow是搜狗公司的C++服务器引擎。

