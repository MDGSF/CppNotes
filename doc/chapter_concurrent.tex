\chapter{并发编程}

\begin{enumerate}
  \item \textbf{并发}(Concurrent):即使只有一个CPU也可以交替执行多个任务。
  \item \textbf{并行}(Parallel):需要有多个CPU支持。
\end{enumerate}

\section{多进程}

进程是操作系统分配资源的最小单元。我们可以使用多个进程的方式来实现并发。

比如nginx就是使用这种模式,nginx启动一个master进程和多个的worker进程,master进程
只有一些简单的逻辑,主要的业务处理逻辑都交给worker进程去处理。这么做的好处也很明
显,如果某个worker进程挂掉了,不会对其他的worker进程有任何影响,master进程会监控
到这种情况,然后重新启动一个worker进程。所以采用多进程的方式来实现,程序还是非常
健壮的。

但是,要是用错了多进程的方式,效率估计就非常的低了。比如说要是客户端的每一个请
求,都新开一个进程,那估计并发的数量是很难超过1000的。

\section{多线程}

线程是操作系统调度的最小单元。多线程是最常用的实现并发的一种方式。进程一运行起来
就至少有一个主线程,如果要用多个线程,需要用创建线程的接口来创建多个线程。多个线
程之间共享进程中的数据。这样如果有多个线程对同一个数据有读写冲突,就需要加锁来保
护数据的一致性。或者使用原子变量。还可以使用消息队列在线程之间共享数据。

下面给出一个C++11中的多线程代码示例:

\begin{verbatim}
// Copyright (c) 2021 by HuangJian
//
// g++ thread1.cpp -lpthread -std=c++11 -o a.out

#include <iostream>
#include <thread>

using namespace std;

int main() {
  auto f = []() { cout << "tid=" << this_thread::get_id() << endl; };
  thread t1(f);
  t1.join();
  return 0;
}
\end{verbatim}

但是,多线程和多进程一样,如果每个客户端请求都新开一个线程的话,并发量也是非常低
的。另外,一个线程如果挂掉了,整个进程都会挂掉,这个问题不知道能不能用捕获异常来
解决。

\section{协程}

C++20里面才正式加入了协程,所以在C++11里面就只能用第三方库了。协程是用户态的线
程,协程的开销远远小于线程,所以协程可以每个客户端请求都新开一个协程,非常好,可
惜C++11用不了。

\section{锁}

\subsection{互斥锁}

\begin{verbatim}
#include <iostream>  // std::cout
#include <mutex>     // std::mutex
#include <thread>    // std::thread

std::mutex mtx;  // mutex for critical section

void print_block(int n, char c) {
  mtx.lock();
  for (int i = 0; i < n; ++i) {
    std::cout << c;
  }
  std::cout << '\n';
  mtx.unlock();
}

int main() {
  std::thread th1(print_block, 50, '*');
  std::thread th2(print_block, 50, '$');
  th1.join();
  th2.join();
  return 0;
}
\end{verbatim}

\subsection{自旋锁}

自旋锁和互斥锁有些区别,自旋锁在无法立即获取到锁的时候,不会立马退出运行态,会一
直处于运行态占用CPU,直到获取到锁为止。这样会更快获取到锁,但是会占用更多的CPU资
源。在你不确定要用互斥锁还是自旋锁的时候,那就使用互斥锁。在现代的操作系统中,互
斥锁的实现很多时候在加锁之前如果获取不到锁,也是会自旋一小会。所以在你需要使用自
旋锁的时候,需要具体测试性能。

\section{原子变量}

\subsection{memory\_order}

\begin{enumerate}
  \item memory\_order\_relaxed
  \item memory\_order\_consume
  \item memory\_order\_acquire
  \item memory\_order\_release
  \item memory\_order\_acq\_rel
  \item memory\_order\_seq\_cst
\end{enumerate}

https://en.cppreference.com/w/cpp/atomic/memory\_order

\section{内存屏障}

\textbf{内存屏障}(英语:Memory barrier),也称内存栅栏,内存栅障,屏障指令等,
是一类同步屏障指令,它使得 CPU 或编译器在对内存进行操作的时候, 严格按照一定的顺
序来执行,也就是说在memory barrier 之前的指令和memory barrier之后的指令不会由于系
统优化等原因而导致乱序。

为啥需要内存屏障呢?大多数现代计算机为了提高性能而采取乱序执行,这使得内存屏障成
为必须。

大多数处理器提供了内存屏障指令:

\begin{enumerate}
  \item 完全内存屏障(full memory barrier)保障了早于屏障的内存读写操作的结果提交
    到内存之后,再执行晚于屏障的读写操作。
  \item 内存读屏障(read memory barrier)仅确保了内存读操作;
  \item 内存写屏障(write memory barrier)仅保证了内存写操作。
\end{enumerate}

https://en.wikipedia.org/wiki/Memory\_barrier \\
https://en.wikipedia.org/wiki/Out-of-order\_execution \\
https://www.kernel.org/doc/Documentation/memory-barriers.txt

