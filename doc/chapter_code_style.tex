\chapter{代码风格}

代码风格分为两类:

\begin{enumerate}
  \item 代码格式的风格,例如括号要放在一行的末尾还是另起一行。
  \item 代码命名的风格,例如函数名要用匈牙利命名还是驼峰命名。
\end{enumerate}

常见的\textbf{代码自动格式化工具}有两个:

\begin{enumerate}
  \item Astyle
  \item clang-format
\end{enumerate}

clang-format配置使用google代码风格:

\begin{verbatim}
clang-format -style=google -dump-config > .clang-format
\end{verbatim}

clang-format会自动查找.clang-format配置文件,当前目录找不到会找父目录。所以可以
在HOME目录下生成一个.clang-format作为默认配置。也可以在项目的根目录下生成一个
.clang-format配置文件作为项目的默认配置。

常见的\textbf{代码命名风格}有三种:

\begin{enumerate}
  \item 匈牙利命名法
  \item 驼峰命名法,CamelCase
  \item 蛇形命名法,snake\_case
\end{enumerate}

建议的命名风格如下:

\begin{verbatim}
#define MAX_PATH_LEN 256 //常量,全大写

int g_sys_flag; // 全局变量,加g_前缀

namespace linux_sys { // 名字空间,全小写
  void get_rlimit_core(); // 函数,全小写
}

class FilePath final // 类名,首字母大写
{
public:
  void set_path(const string& str); // 函数,全小写

private:
  string m_path; // 成员变量,m_前缀
  int m_level; // 成员变量,m_前缀
};
\end{verbatim}
