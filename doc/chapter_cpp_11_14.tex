\chapter{C++11/14介绍}

\section{C++14}

\subsection{Binary literals}

可以在代码中使用二进制数值。

\begin{verbatim}
// clang binary_literals.cpp -lstdc++
#include <iostream>
using namespace std;
int main() {
  int a = 0b110;  // 6
  cout << "a = " << a << endl;
  int b = 0b1111'1111; // 255
  cout << "b = " << b << endl;
  return 0;
}
\end{verbatim}

\subsection{Generic lambda expressions}

在 lambda 的参数中可以使用泛型。

\begin{verbatim}
auto identity = [](auto x) { return x; };
int three = identity(3);
string foo = identity("foo");
\end{verbatim}

\subsection{Lambda capture initializers}

在 lambda 声明的时候就计算好参数中的表达式的值。

\begin{verbatim}
int factory(int i) { return i * 10; }
auto f = [x = factory(2)] { return x; };
cout << "f() = " << f() << endl; // 20
\end{verbatim}

\begin{verbatim}
auto generator = [x = 0]() mutable { return x++; };
auto a = generator();
auto b = generator();
auto c = generator();
cout << "a = " << a << endl; // 0
cout << "b = " << b << endl; // 1
cout << "c = " << c << endl; // 2
\end{verbatim}

可以把 unique\_ptr move 到 lambda 中。

\begin{verbatim}
auto p = std::make_unique<int>(1);
// auto task = [=] { *p = 5; }; // error
auto task = [p = std::move(p)] { *p = 5; }; // ok
\end{verbatim}

捕获的变量可以重命名。

\begin{verbatim}
auto x = 1;
auto f = [& r = x, x = x * 10] {
  ++r;
  return r + x;
};
cout << "f() = " << f() << endl; // 12
\end{verbatim}

\subsection{Return type deduction}

返回值自动推断。

\begin{verbatim}
auto f(int i) { return i; }

template <typename T>
auto& f(T& t) {
  return t;
}

auto h = [](auto& x) -> auto& { return f(x); };
\end{verbatim}

\subsection{decltype auto}

decltype(auto) 和 auto 很像,但是有些区别。decltype(auto) 会保留 const 和引用,
而 auto 不会。

\begin{verbatim}
const int x = 0;
auto x1 = x;            // int
decltype(auto) x2 = x;  // const int

int y = 0;
int& y1 = y;
auto y2 = y1;            // int
decltype(auto) y3 = y1;  // int&

int&& z = 0;
auto z1 = std::move(z);            // int
decltype(auto) z2 = std::move(z);  // int&&
\end{verbatim}

作为返回值也是一样的。

\begin{verbatim}
// Return type is `int`.
auto f(const int& i) { return i; }

// Return type is `const int&`.
decltype(auto) g(const int& i) { return i; }
\end{verbatim}

\subsection{constexpr}

constexpr函数在C++14中支持更多的语法。

\begin{verbatim}
constexpr int factorial(int n) {
  if (n <= 1) {
    return 1;
  } else {
    return n * factorial(n - 1);
  }
}
\end{verbatim}

\subsection{variable templates}

\begin{verbatim}
template <class T>
constexpr T pi = T(3.1415926535897932385);

template <class T>
constexpr T e = T(2.7182818284590452353);
\end{verbatim}

\subsection{deprecated attribute}

被 deprecated 标记的函数如果被调用了,编译的时候会有警告。

\begin{verbatim}
[[deprecated]]
void old_method() {}

[[deprecated("Use new_method instead")]]
void legacy_method() {}
\end{verbatim}

\section{C++11}

C++11新增的东西很多,建议阅读<<C++ Primer>>和罗剑锋的C++实战笔记。 \\
下面是一个C++11的例子。

\begin{verbatim}
namespace demo_cpp11 {  // 命名空间,全小写

// using 别名
// typedef unsigned int uint_t; // 等价的 typedef
using uint_t = unsigned int;

// 常量,全大写
#define MAX_PATH_LEN 256

// 常量,全大写
const int BUFF_SIZE = 4096;

// 全局变量用 g_ 作为前缀
int g_count = 0;

/*
@brief DemoCpp11 C++11最佳实践示范类。
1. 给类添加 final,防止被继承。
2. 现在每个类都有 6 个默认函数:
 2.1 默认构造函数
 2.2 默认析构函数
 2.3 默认拷贝构造函数
 2.4 默认拷贝赋值函数
 2.5 默认转移构造函数
 2.6 默认转移赋值函数
可以通过 `= default` 来明确使用默认实现。
可以通过 `= delete` 来明确禁用默认实现。
3. 使用 explicit 来防止一些隐式转换。
4. 可以使用代理构造函数来简化代码。
5. 成员变量声明的时候可以直接初始化。
6. 使用 const 来修饰常函数。
7. 使用 using 来定义类型别名。
8. 类名使用 CamelCase,函数和变量用 snake_case,成员变量加 m_ 前缀
*/
class DemoCpp11 final {  // 添加 final,类就没法被继承了。
 public:
  using this_type = DemoCpp11;      // 给自己也起个别名
  using string_type = std::string;  // 字符串类型别名
  using set_type = std::set<int>;   // 集合类型别名

 public:
  DemoCpp11() = default;   // 明确告诉编译器,使用默认实现
  ~DemoCpp11() = default;  // 明确告诉编译器,使用默认实现

  DemoCpp11(const DemoCpp11&) = delete;             // 禁止拷贝构造函数
  DemoCpp11& operator=(const DemoCpp11&) = delete;  // 禁止拷贝赋值函数

  DemoCpp11(DemoCpp11&& s) = default;             // 显式转移构造函数
  DemoCpp11& operator=(DemoCpp11&& s) = default;  // 显式转移赋值函数

  explicit DemoCpp11(int id) : m_id(id) {}  // 显式单参构造函数
  explicit operator bool();                 // 显式转型为bool

  // 代理构造函数
  explicit DemoCpp11(const std::string& id)  // 字符串参数构造函数
      : DemoCpp11(stoi(id)) {}  // 转换成整数,再委托给另一个构造函数

  // 添加 const 修饰常函数
  // 函数名,全部小写
  int get_id() const { return m_id; }

  // 函数名,全部小写
  void show_demo() const {
    // 局部变量,全部小写
    int local_variable = 123;
    std::cout << "local_variable = " << local_variable << std::endl;
    std::cout << "m_id = " << m_id << std::endl;
    std::cout << "m_name = " << m_name << std::endl;
  }

 private:
  /*
  成员变量,用 m_ 作为前缀,全部小写
  */
  int m_id = 0;                      // 整数成员,赋值初始化
  std::string m_name = "democpp11";  // 字符串成员,赋值初始化
  std::vector<int> m_friends{1, 2, 3};  // 容器成员,使用花括号的初始化列表
};

} // namespace demo_cpp11
\end{verbatim}

